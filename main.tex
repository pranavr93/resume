%%%%%%%%%%%%%%%%%%%%%%%%%%%%%%%%%%%%%%%
% LaTeX Template
% Version 1.1 (30/4/2014)
%
% Original author:
% Debarghya Das (http://debarghyadas.com)
% New author:
% Pranav Ramarao
\documentclass[]{resume-openfont}
\begin{document} 

%%%%%%%%%%%%%%%%%%%%%%%%%%%%%%%%%%%%%%
%
%     LAST UPDATED DATE
%
%%%%%%%%%%%%%%%%%%%%%%%%%%%%%%%%%%%%%%
\lastupdated

%%%%%%%%%%%%%%%%%%%%%%%%%%%%%%%%%%%%%%
%
%     TITLE NAME
%
%%%%%%%%%%%%%%%%%%%%%%%%%%%%%%%%%%%%%%
\namesection{Pranav}{Ramarao}{ 
% \urlstyle{same}\url{http://linkedin.com/pranavramarao} \\
\Letter \href{mailto:pranavr@umich.edu}{ pranavr@umich.edu } \textbullet{} { \Mobilefone {  +1 (734)-680-4390}} \\
Full Time Roles \textbullet{} Software Engineering Position \textbullet{} Graduating Dec 2017\\
}
%%%%%%%%%%%%%%%%%%%%%%%%%%%%%%%%%%%%%%
%
%     COLUMN ONE
%
%%%%%%%%%%%%%%%%%%%%%%%%%%%%%%%%%%%%%%
\begin{minipage}[t]{0.25\textwidth} 

%%%%%%%%%%%%%%%%%%%%%%%%%%%%%%%%%%%%%%
%     EDUCATION
%%%%%%%%%%%%%%%%%%%%%%%%%%%%%%%%%%%%%%

\section{Education} 

\subsection{University of Michigan}
\descript{MS in Computer Science}
\location{Exp Dec 2017 | Ann Arbor
\\ GPA: 3.79}
\sectionsep
\textbullet{} TA for EECS 281 (Data Structures and Algorithms)\\
\textbullet{} Taught 120+ students\\
\textbullet{} Outstanding TA award
\sectionsep

\subsection{BITS Pilani}
\descript{BE in Computer Science}
\location{May 2015 | Hyderabad \\ GPA: 9.29}
\sectionsep

%%%%%%%%%%%%%%%%%%%%%%%%%%%%%%%%%%%%%%
%     SKILLS
%%%%%%%%%%%%%%%%%%%%%%%%%%%%%%%%%%%%%%
\section{Skills}
\subsection{Focus Areas}
\textbullet{} Algorithms \\
\textbullet{} Information Retrieval \\
\textbullet{} Machine Learning \\
\textbullet{} Distributed Systems \\
\textbullet{} Software Engineering\\
\sectionsep

\subsection{Programming}
\location{Proficient in:}
C++ \textbullet{}   C\# \textbullet{} Python \textbullet{} Java \textbullet{} SQL \textbullet{} Git \textbullet{} Shell\\
\location{IDE:}
Visual Studio \textbullet{} Xcode\textbullet{} Eclipse \\

\sectionsep

\subsection{Online Judges}
Codechef rating: 1958 \\
Codeforces rating: 1701 \\
\textbullet{} Solved 500+ problems \\
\textbullet{} Won several awards in coding competitions
\sectionsep

%%%%%%%%%%%%%%%%%%%%%%%%%%%%%%%%%%%%%%
%     LINKS
%%%%%%%%%%%%%%%%%%%%%%%%%%%%%%%%%%%%%%

\section{Links} 
Github:// \href{https://github.com/pranavr93}{\custombold{pranavr93}} \\
LinkedIn://  \href{https://www.linkedin.com/in/pranavramarao}{\custombold{pranavramarao}} \\
\sectionsep

%%%%%%%%%%%%%%%%%%%%%%%%%%%%%%%%%%%%%%
%     REFERENCES
%%%%%%%%%%%%%%%%%%%%%%%%%%%%%%%%%%%%%%

\section{References}
\descript{B. Ashok}
Senior Director \\
Microsoft Research India \\
bash@microsoft.com \\

\sectionsep
\descript{Suresh Parthasarathy}
Senior Research Engineer \\
Microsoft Research, India \\
supartha@microsoft.com

% \sectionsep
% \descript{Dr. David Paoletti}
% EECS 281 Professor  \\
% University of Michigan \\
% paoletti@umich.com
%%%%%%%%%%%%%%%%%%%%%%%%%%%%%%%%%%%%%%
%
%     COLUMN TWO
%
%%%%%%%%%%%%%%%%%%%%%%%%%%%%%%%%%%%%%%

\end{minipage} 
\hfill
\begin{minipage}[t]{0.74\textwidth} 

%%%%%%%%%%%%%%%%%%%%%%%%%%%%%%%%%%%%%%
%     EXPERIENCE
%%%%%%%%%%%%%%%%%%%%%%%%%%%%%%%%%%%%%%

\section{Experience}

\runsubsection{Google}
\descript{| Software Engineering Intern }
\location{June 2017 – August 2017 | Sunnyvale, California}
\vspace{\topsep} % Hacky fix for awkward extra vertical space
\begin{tightemize}
\item Worked in the storage platforms team on bringing additional analytics metrics on HDD drives, such as throughput, tail latencies and DMA histograms into the current pipeline.
\item Anomalies based on the above metrics are detected in drives across the Google fleet and alerts are raised for repair strategies at software/hardware level.
\end{tightemize}
\sectionsep

% \runsubsection{University of Michigan}
% \descript{| Graduate Student Instructor }
% \location{Sept 2016 – May 2017 | Ann Arbor}
% \begin{tightemize}
% \item Teach EECS 281 (Algorithms and Data Structures) for undergrad students
% \item Handle discussion sections (120+ students overall), office hours (assisted 200+ students), helped set lab assignments, projects and exams.
% \end{tightemize}
% \sectionsep

\runsubsection{Microsoft Research}
\descript{| Research Fellow }
\location{July 2015 – July 2016 | Bangalore, India}
\begin{tightemize}
\item Engaged in end to end development of a new e-mail client (Email Insights) that supported powerful context based search, auto-completion, spell correction and fuzzy contact search. 
\item Demoed it at TechFest in Microsoft HQ - selected for Garage (public) release
\item Public release and press links: 
\textbf{\href{http://www.zdnet.com/product/email-insights/}{Zdnet, }}
\textbf{\href{https://www.microsoft.com/en-us/garage/profiles/email-insights/}{Microsoft Garage, }}
\textbf{\href{http://www.pcworld.com/article/3170146/windows/microsofts-email-insights-finally-adds-some-useful-search-smarts-to-outlook.html}{Pcworld}}

\item The work was published in ACM - SIGIR, 2016  \textbf{\href{http://dl.acm.org/citation.cfm?id=2911451.2911458}{(InLook: Revisiting Email Search Experience)}}
\end{tightemize}
\sectionsep

\runsubsection{Microsoft Research}
\descript{| Software Engineering Intern}
\location{Jan 2015 – May 2015 | Bangalore, India}
\begin{tightemize}
\item Worked on ‘Debug Advisor’ - an information assistant that provided developers contextual information during bug resolution based on past data from similar solved bugs. 
\item Made use of Titan (graph database) and developed high performance algorithms for indexing 30 million records in a few hours and supported quick query on it.
\end{tightemize}
\sectionsep

% \runsubsection{Google Summer of Code}
% \descript{| Student Developer}
% \location{June 2015 – Sept 2015 | Bangalore, India}
% \begin{tightemize}
% \item Closely interacted with the Mono project (Xamarin) team and built a code-visualization add-in in Xamarin Studio.
% \item Made use of a language service (NRefactory6 - Roslyn) and developed layout algorithms for the same.
% \end{tightemize}
% \sectionsep
\runsubsection{Gradbusters.com}
\descript{| Co-Founder}
\begin{tightemize}
\item Lead a team of engineers in building a data centric platform that helps students aspiring to do their graduate studies in the US. 
\item The website provides intelligent tools that makes predictions and recommendations based on past data.
\item The site had 100k+ visits and 7000 registered users within 6 months of launch.
\end{tightemize}
\sectionsep

%%%%%%%%%%%%%%%%%%%%%%%%%%%%%%%%%%%%%%
%     PROJECTS
%%%%%%%%%%%%%%%%%%%%%%%%%%%%%%%%%%%%%%

\section{Projects}
% \runsubsection{Distributed Sharded Key Value Store}
\descript{Distributed Sharded Key Value Store}
\vspace{-7pt}\justify Built a sharded key value store backed by Paxos in C++14. The system supports linearizable consistency. Consistent hashing was used for key distribution. The KV-store can handle concurrent requests, failures of replicas and addition of new shards. \textbf{\href{https://github.com/pranavr93/sharded_key_value_store}{Github}}
\sectionsep

\descript{Piazza BOT}
\vspace{-7pt}\justify Developed a smart bot for piazza- Jarvis that aimed to improve the efficiency of instructors and experience for students. The bot performed duplicate question detection, automated weekly FAQ generation and provided a smarter search.  \textbf{\href{https://github.com/pranavr93/piazza_bot}{Github}}
\sectionsep

\descript{Google Summer of Code '15}
\vspace{-7pt}\justify Closely interacted with the Mono project (Xamarin) team and built a code-visualization add-in in Xamarin Studio. Made use of a language service (NRefactory6 - Roslyn) and developed layout algorithms for the same. \textbf{\href{https://github.com/pranavr93/MDClassDiagram}{Github}}
\sectionsep

% %%%%%%%%%%%%%%%%%%%%%%%%%%%%%%%%%%%%%%
% %     PATENTS AND PUBLICATIONS
% %%%%%%%%%%%%%%%%%%%%%%%%%%%%%%%%%%%%%%

% \section{Patents and Publications}

% \begin{tightemize}
% \item Pranav Ramarao, et al. “Contextual windows for general purpose applications” (Patent pending)
% \item Pranav Ramarao, et al. “InLook: Rethinking Emails”, 39th International ACM SIGIR Conference on Research and Development in IR, Pisa, Italy
% \textbf{\href{http://dl.acm.org/citation.cfm?id=2911451.2911458}{Link}}
% \end{tightemize}
% \sectionsep

% %%%%%%%%%%%%%%%%%%%%%%%%%%%%%%%%%%%%%%
% %     PATENTS AND PUBLICATIONS
% %%%%%%%%%%%%%%%%%%%%%%%%%%%%%%%%%%%%%%

% \section{Patents and Publications}
% \begin{tightemize}
% \item Pranav Ramarao, Suresh Iyengar, C.Pushkar, U. Raghavendra, B.Ashok, “Contextual windows for general purpose applications”, Application No MS \#359952.01/41827-8143IN (Patent pending)
% \item Pranav Ramarao, Suresh Iyengar, C.Pushkar, U. Raghavendra, B.Ashok, “InLook: Rethinking Emails”, 39th International ACM SIGIR Conference on Research and Development in Information Retrieval, Pisa, Italy
% \textbf{\href{http://dl.acm.org/citation.cfm?id=2911451.2911458}{Link}}
% \item Pranav Ramarao, K Muthukumaran, D. Siddharth, N L Bhanu Murthy “Impact of Bug Reporter’s Reputation on Bug-fix Times”, International Conference on Information Systems Engineering (ICISE2016), Los Angeles, USA
% \textbf{\href{http://ieeexplore.ieee.org/xpls/abs_all.jsp?arnumber=7486274}{Link}}
% \end{tightemize}
% \sectionsep

%%%%%%%%%%%%%%%%%%%%%%%%%%%%%%%%%%%%%%
%     ACCOMPLISHMENTS
%%%%%%%%%%%%%%%%%%%%%%%%%%%%%%%%%%%%%%

\section{ACCOMPLISHMENTS} 
\begin{tabular}{rll}
2016	     & Won the company-wide Hackathon at Microsoft in the Internet Category \\
2015 		& Pending Patent: "Contextual windows for General Purpose Applications"\\
2014	     & Winner of Google Codejam India as part of the GDG conference  \\
\end{tabular}
\sectionsep

\end{minipage} 
\end{document}  \documentclass[]{article}